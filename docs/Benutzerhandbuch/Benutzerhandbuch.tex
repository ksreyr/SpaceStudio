\documentclass[fontsize=12pt,paper=a4,twoside]{scrartcl}
\usepackage{longtable} 
\usepackage{graphicx}
% SWP-Präambel
% C 2003-2017 Sebastian Offermann, Rainer Koschke, Karsten Hölscher
% In Zeilen 40 und 41 sind jeweils die aktuellen Daten einzutragen

\usepackage[utf8]{inputenc}     % Kodierung der Tex-Datei
\usepackage[T1]{fontenc}        % Korrekte Ausgabe von Sonderzeichen (Umlaute)
\usepackage[ngerman]{babel}     % Deutsche Einstellungen [ab \begin{document}]

\usepackage{bibgerm}            % Bibliographie
\usepackage{fancyhdr}           % obere Seitenränder gestalten
\usepackage{float}              % Floats Objekte mit [H] festsetzen
\usepackage{graphicx}           % Graphiken als jpg, png etc. einbinden
\usepackage{moreverb}           % zusätzliche verbatim-Umgebungen
\usepackage{pdflscape}          % PDF-Support für landscape
\usepackage[final]{pdfpages}    % Externe PDFs einbinden
\usepackage{stmaryrd}           % zusätzliche Symbole
\usepackage{supertabular}       % Tabellen über Seitenränder hinaus
\usepackage{tabularx}           % Tabellen mit vorgegebener Breite
\usepackage{url}                % setzt URLs schön mit \url{http://bla.laber.com/~mypage}

%%% Die Reihenfolge der folgenden Pakete muss beibehalten werden:
%%% varioref, hyperref, cleveref, bookmark
% Verweise innerhalb des Dokuments schick mit " ... auf Seite ... "
% automatisch versehen. Dazu \vref{labelname} benutzen
\usepackage[ngerman]{varioref}  % [vor hyperref für korrekte Verweise]
\usepackage[colorlinks=true, pdfstartview=FitV, linkcolor=blue,
            citecolor=blue, urlcolor=blue, hyperfigures=true,
            pdftex=true]{hyperref} % [vor bookmark wegen der Optionen]
\usepackage[ngerman]{cleveref}
\usepackage{bookmark}

\hyphenation{Arbeits-paket}     % Trennungsregeln

%%% Definitionen
\newcommand{\grad}{\ensuremath{^{\circ}} }
\renewcommand{\strut}{\vrule width 0pt height5mm depth2mm}
\newcommand{\gq}[1]{\glqq{}#1\grqq{}}

%%% Semesterkonstanten
\newboolean{langversion} %Deklaration
\setboolean{langversion}{true} %Zuweisung ist 'false' für Blockkurs
\newcommand{\jahr}[1]{2020} %2017/2018

% erstes Argument: SWP-2, zweites SWP-1
\newcommand{\highlight}[1]{\textcolor{blue}{\textbf{#1}}}
\newcommand{\variante}[2]{\ifthenelse{\boolean{langversion}}{#1}{#2}}
\newcommand{\nurlangversion}[0]%
    {\variante{\highlight{}}%Muss in SWP-2 ausgefüllt werden}}%
              {\highlight{Entfällt in SWP-1}}}
\newcommand{\swp}[0]{Software-Projekt \variante{2}{1}}
\newcommand{\semester}[0]{SoSe \jahr}

%%% Formatierungsanpassungen
% Damit Latex nicht zu lange Zeilen produziert:
\sloppy
%Uneinheitlicher unterer Seitenrand:
%\raggedbottom

% Kein Erstzeileneinzug beim Absatzanfang
% Sieht aber nur gut aus, wenn man zwischen Absätzen viel Platz einbaut
\setlength{\parindent}{0ex}

% Abstand zwischen zwei Absätzen
\setlength{\parskip}{1ex}

% Seitenränder für Korrekturen verändern
\addtolength{\evensidemargin}{-1cm}
\addtolength{\oddsidemargin}{1cm}

\bibliographystyle{gerapali}

% 1. Parameter: Euer/Eure TutorIn, z. B. {Kim Harrison}
% 2. Parameter: Abgabedatum, z. B. {05. April 2063}
% 3. Parameter: Versionsnummer, z. B. {1.1}
% 4.-9. Parameter: jeweils Name und (Uni-)Email-Adresse jedes 
%                 Gruppenmitglieds; mit einem & getrennt, z. B.
% {Robin Cowl & roco@tzi.de}
% Besteht die Gruppe aus weniger als 6 Personen, so werden die 
% übrigen Parameter leer gelassen: {}
\newcommand \swpdocument[9] {
% Lustige Header auf den Seiten
  \pagestyle{fancy}
  \setlength{\headheight}{70.55003pt}
  \fancyhead{}
  \fancyhead[LO,RE]{\swp{}\\%
                    \semester{}\\%
                    \documentTitle}
  \fancyhead[LE,RO]{Seite \thepage\\%
                    \slshape \leftmark\\%
                    \slshape \rightmark}

% Lustige Header nur auf dieser Seite (Titelseite)
  \thispagestyle{fancy}
  \fancyhead[LO,RE]{ }
  \fancyhead[LE,RO]{Universität Bremen\\%
                    FB 3 -- Informatik\\%
                    Dr. Karsten Hölscher\\%
                    TutorIn: #1}
  \fancyfoot[C]{}

% Start Titelseite
  \vspace{3cm}
  \begin{minipage}[H]{\textwidth}
    \begin{center}
      \bfseries \Large \swp{} -- \semester{}\\
      \smallskip
      \small VAK 03-BA-901.02\\
      \vspace{3cm}
    \end{center}
  \end{minipage}
  \begin{minipage}[H]{\textwidth}
    \begin{center}
      \vspace{1cm}
      \bfseries \Large \documentTitle\\
      \vfill
    \end{center}
  \end{minipage}
  \vfill
  \begin{minipage}[H]{\textwidth}
    \begin{center}
      \sffamily
      \begin{tabular}{lr}
        #4 \\
        #5 \\
        #6 \\
        #7 \\
        #8 \\
        #9 \\
      \end{tabular}
      \\[22mm]
      \itshape Abgabe: #2 --- Version #3 \\ ~
    \end{center}
  \end{minipage}
% Ende Titelseite

% Start Inhaltsverzeichnis
\newpage
  \thispagestyle{fancy}
  \fancyhead{}
  \fancyhead[LO,RE]{\swp{}\\%
                    \semester{}\\%
                    \documentTitle}
  \fancyhead[LE,RO]{Seite \thepage\\%
                    \slshape \leftmark\\~}
  \fancyfoot{}
  \renewcommand{\headrulewidth}{0.4pt}
  \tableofcontents
% Ende Inhaltsverzeichnis

% Header für alle weiteren Seiten
\newpage
  \fancyhead[LE,RO]{Seite \thepage\\%
                    \slshape \leftmark\\%
                    \slshape \rightmark}

}



%
% Und jetzt geht das Dokument los....
%
\begin{document}
\newcommand\documentTitle{Benutzerhandbuch}
 %\begin{minipage}[b]{\textwidth}
 \vspace{1mm}
 \begin{figure}[!b]
  \centering
  \includegraphics[width=0.5\textwidth]{pics/SpaceStudioLogo.png}\\
\end{figure}

\swpdocument{Dr. Karsten Hölscher}{02. August 2020}{1.1}%
            {Clara Maria Odinius & odinius@uni-bremen.de}%
            {Habib Mergan & habib1@uni-bremen.de}%
            {Kevin Santiago Rey Rodriguez & kev\_rey@uni-bremen.de}%
            {Liam Hurwitz & hurwitz@uni-bremen.de}%
            {Mehmet Ali Baykara & baykara@uni-bremen.de}%
            {Miguel Alejandro Caceres Pedraza & mcaceres@uni-bremen.de}%

%%%%%%%%%%%%%%%%%%%%%%%%%%%%%%%%%%%%%%%%%%%%%%%%%%%%%%%%%%%%%%%%%%%%%%%%



%%%%%%%%%%%%%%%%%%%%%%%%%%%%%%%%%%%%%%%%%%%%%%%%%%%%%%%%%%%%%%%%%%%%%%%%
\section{Einführung}

Willkommen beim Spacestudio Game!\\
Dieses Benutzerhandbuch beschreibt, wie die Software bzw. das Spiel gespielt wird und dient somit als Anleitung für das Programm.
Lesen Sie sich dieses Handbuch aufmerksam durch bevor Sie die Anwendung starten, um alle Funktionalitäten des Spiels kennen zu lernen und zu verstehen.

Dieses Spiel ist nach dem Vorbild des bekannten Spiels Faster Than Light entwickelt worden, daher werden Sie viele Parallelen zwischen diesen beiden Spielen feststellen können.

%%%%%%%%%%%%%%%%%%%%%%%%%%%%%%%%%%%%%%%%%%%%%%%%%%%%%%%%%%%%%%%%%%%%%%%%
\subsection{Zweck}

Dieses Benutzerhandbuch ist im Rahmen von ReSWP 2 SoSe2020  der Gruppe SpaceStudio geschrieben.
Der Zweck dieses Handbuches besteht darin, eine Anleitung zu dem entwickleten Spiel bereit zu stellen, damit sowohl das Spiel selbst, als auch unsere Gedanken und Ideen bezüglich der vielen einzelnen Elemente nachvollzogen werden können.

%%%%%%%%%%%%%%%%%%%%%%%%%%%%%%%%%%%%%%%%%%%%%%%%%%%%%%%%%%%%%%%%%%%%%%%%

\section{Installation}
In diesem Abschnitt wird beschieben, wie das Programm installiert wird und entsprechende Systemvoraussetzungen aufgelistet.

\subsection{Systemvoraussetzungen}
1. Java 11 oder höher \\
2. Windows 10, macOS, Linux \\
3. Gradle 5.2 oder höher \\
4. Spring Boot 2.0.3.RELEASE \\
5.  Spring Framework 5.0.7.RELEASE or höher. \\
6. 1920X1080 Auflösung
 

\subsection{Installationsschritte}
Folgende  Schritte sind für Installation benötigt. \\
1. Nach dem Sie alle benötigte tools, die in Systemvoraussetzungen aufgelistet wurden, installiert haben \\
2. Starten Sie der Server und warten Sie bis der Server hochfährt (ca 1 Minute)\\
3. Wenn der Server gestartet erfolreich gestartet wurde, starten Sie den Client.

 


%%%%%%%%%%%%%%%%%%%%%%%%%%%%%%%%%%%%%%%%%%%%%%%%%%%%%%%%%%%%%%%%%%%%%%%%

\section{Start}

Startet der Client, dann zeigt sich sich der Registrierungs und Login Screen.
Registrieren Sie sich über die Eingabefelder auf der rechten Seite. Dazu wählen Sie einen Benutzernamen und Passwort, welches zwei Mal wiederholt eingegeben werden muss, um fehlerhafte Eingaben zu vermeiden.
War die Registrierung erfolgreich, bekommen Sie die Rückmeldung Successfully created, anderen falls erscheint Password does not match oder sofern der Benutzname schon vergeben wurde Name already
registered, try another one.\\
Nach erfolgreicher Registrierung geben Sie ihre Benutzernamen und Passwort in die linken Eingabefelder ein, ist auch der Login erfolgreich erscheint die Rückmeldung login... .\\

Hinweis: Gib es Probleme mit der Registrierung, empfehlen wir den Server erneut zu starten.

\subsection{Anmeldung}
\begin{figure}[htp]
\centering
	\includegraphics[width=1.00\linewidth]{pics/StartScreen01.png}
	\caption{Anmeldung: Registrierung und Login}
	\label{fig1}
\end{figure}

Der Start Screenerstellt erzeugt ein Profil mit welchem sich der Benutzer jeder Zeit wieder beim Spiel einloggen kann, um auf die gespeicherten Spieldaten wieder zurückgreifen zu können. \\
Darüber hinaus besteht die Möglichkeit eine Netwerkadresse einzugeben, falls zu einem Späteren Zeitpunkt der Multiplayer-Modus gewählt wird, dazu mehr im Kapitel Multiplayer-Modus.\\
In der unteren linken Ecke des Fensters befinden sich die zwei Button mute und exit. Über den Mute-Button lässt sich die Hintergrundmusik an- und ausschalten. 
Über den Exit-Button kann das Spiel ohne weiteres beendet werden.

\newpage

%%%%%%%%%%%%%%%%%%%%%%%%%%%%%%%%%%%%%%%%%%%%%%%%%%%%%%%%%%%%%%%%%%%%%%%%
\subsection{Menu}

Nach dem Registrierungs und Login Screen folgt das Menü:\\
Zu sehen sind die drei Button New Game, Options und Exit. Sofern vorher schon ein Spiel begonnen wurde, erscheint auch der Button Continue.\\

\begin{figure}[htp]
	\centering
	\includegraphics[width=1.00\linewidth]{pics/menuscreen.png}
	\caption{Menu}
	\label{fig1}

\end{figure}

Über New Game kann ein neues Spiel begonnen werden. \\
Continue ermöglicht es, ein zuvor begonnenes Spiel weiter zu Spielen, da jeder Spielstand gespeichert werden kann.\\
Options ist derzeit nicht verfügbar und kann für persönliche Anpassungen implementiert werden. \\
Exit beendet das Programm umgehend, ohne zur Startseite zurück zu gelangen.\\

\newpage
%%%%%%%%%%%%%%%%%%%%%%%%%%%%%%%%%%%%%%%%%%%%%%%%%%%%%%%%%%%%%%%%%%%%%%%%
\subsection{New Game}

Sofern im Menu Screen new Game ausgewählt wurde, folgt der New Game Screen.\\
Hier ergeben sich die Möglichkeiten ein Single Player Spiel oder ein Multiplayerspiel zu starten, oder aber zurück zum Menü zu gelangen.\\

\begin{figure}[htp]
	\centering
	\includegraphics[width=1.00\linewidth]{pics/gamemodescreen.png}
	\caption{Spielmodus}
	%\label{fig1}
\end{figure}

\subsection{Single Player}

Hier sehen Sie den Screen, der erscheint, wenn Sie Single Player im New Game Screen gewählt haben:\\
Mittig werden die Raumschiffe angezeigt, die in dem Universum existieren, derzeit verfügt der Spieler aber nur über die Möglichkeit mit dem blauen Schiff zu spielen.\\
Über previous und next lassen sich die Raumschiffe der Zukunft anzeigen. Über den Button Show rooms unterhalb des Raumschiffs, erscheinen die Sektionen, sowie dessen Ausstattung
und Besatzung.

\begin{figure}[htp]
	\centering
	\includegraphics[width=1.00\linewidth]{pics/SinglePlayer01.png}
	\caption{Single Player}
	%\label{fig1}
\end{figure}

Die Anzeige am unteren Bildschirmrand zeigt die zugehörigen Crew Member und dessen Namen, sowie über welche Systeme das Raumschiff verfügt.\\
Links neben dem Start Button kann eine weitere Auswahl zwischen Easy und Normal getroffen werden, diese Modi entscheiden über den Schwierigkeitsgrad des Spiels.\\
Über Start beginnt das Spiel mit der Ausgefählen Spielstufe.
Der Button Back oben links in der Ecke des Fensters, leitet zurück zum New Game Screen.\\
Im Single Player Modus werden die Gegner, auf die der Spiele im laufe des Spiels trifft durch eine KI realisiert.

\newpage
%%%%%%%%%%%%%%%%%%%%%%%%%%%%%%%%%%%%%%%%%%%%%%%%%%%%%%%%%%%%%%%%%%%%%%%%
\subsection{Multiplayer}

Hier sehen Sie den Screen, der erscheint, wenn Sie  Multiplayer im New Game Screen gewählt haben:\\
Dieser Screen unterscheidet sich in sofern vom Singlaplayer Screen, dass eine zusätzliche Anzeige erscheint, der Sie entnehmen können, wie viele weitere Spieler online sind.\\
Damit weiter Spieler erscheinen müssen diese im Login Screen die selber Netzwerkadresse angegeben haben/auf dem gleichen Server angemeldet sein und ebenfalls den Multiplayer Modus augewählt haben.\\
Außerdem entfällt die Wahl der Spielstufe, da der Gegner nun nicht mehr durch den Computer ersetzt werden muss.

\begin{figure}[htp]
	\centering
	\includegraphics[width=1.00\linewidth]{pics/MultiPlayer01.png}
	\caption{Multiplayer}
	\label{fig1}
\end{figure}

Für den Fall, dass keine weiteren Spieler gefunden werden, erscheint ein Pop up Fenster das die Möglichkeiten bietet, über Try again erneut nach anderen Spielern zu suchen
oder sich über Play with AI doch zu einem Spiel im Single Player Modus umzuentscheiden.\\
 
\newpage
%%%%%%%%%%%%%%%%%%%%%%%%%%%%%%%%%%%%%%%%%%%%%%%%%%%%%%%%%%%%%%%%%%%%%%%%
\section{Univesum}

Nach klicken des Start Buttons des vorherigen Screens, gelangen Sie zur Map des Universums.\\
In diesem Beispiel sehen Sie die Map des Easy Universum. Das Raumschiff befindet sich auf dem Start Planeten unten links. Zu sehen sind vier weitere Planeten sowie eine Shoppingmall.\\
All diese Stationen können angeflogen werden. Wird die Shoppingmall angeflogen, gelangt man zum Shop Screen, dazu mehr im Kapitel Shopping.\\
\begin{figure}[htp]
	\centering
	\includegraphics[width=1.00\linewidth]{pics/universeEasyP1.png}
	\caption{Map des Universums}
	\label{fig1}
\end{figure}

Wird eines der weiteren vier Planeten angeflogen (der Startplanet birgt kein Ereignis), dann erscheint ein
Pop up Fenster mit einer Information zu diesem Planeten. Die Optionen sind, per jump diesen anzufliegen und zu landen oder über back zurück zur Map zu gelangen. Der Planet der am weitesten rechts liegt, kann jedoch nur als letztes angeflogen werden, wenn der Spiele die Ereignisse der vorherigen Planeten gemeistert hat, denn auf diesem wartet der Endgegner auf einen und nur wenn auch dieser besiegt wurde, ist das Spiel endgültig gewonnen. Zu den Stationen mehr im Kapitel Kampf.\\
Wird die Shopping Mall angeflogen, gelangt das Raumschiff zum Shop Screen, in dem er die Möglichkeit hat Ressourcen für sein Raumschiff einzukaufen, dazu mehr im Kapitel Shopping.\\

\begin{figure}[htp]
	\centering
	\includegraphics[width=1.00\linewidth]{pics/infoPlanet2.png} %Bild austauschen
	\caption{Info zu Planet}
	\label{fig1}
\end{figure}

Ein Planet, der bereits besucht wurde verfärbt sich dunkel:

\begin{figure}[htp]
	\centering
	\includegraphics[width=0.30\linewidth]{pics/visitedPlanet.png}
	\caption{visited Planet}
	\label{fig1}
\end{figure}

Oben rechts im Fenster ist der Button Save game zu sehen, dieser ermöglicht das Speichern des Spiels, zu jenem Zeipunkt, wenn sich das Raumschiff wieder auf der Map befindet.

%%%%%%%%%%%%%%%%%%%%%%%%%%%%%%%%%%%%%%%%%%%%%%%%%%%%%%%%%%%%%%%%%%%%%%%%
\section{Einen Planeten anfliegen}
Haben Sie sich dazu entschieden per Jump, einen Planeten azufliegen, gelangen Sie zum Travelscreen.
%Bild Travelscreen einfügen
Ein Infofenster erscheint und beschreibt den Planeten, daraufhin muss sich der Spieler je nach dem, wie er die Lage einschätzt, zwischen Explore und Leave entscheiden. Leave führt zurück zur Map und Explore führt die Reise fort.\\
Nach einer kurzen Weile tritt ein zufälliges Ereignis ein, dass sich positiv ode Negativ auf den Spieler auswirken kann. In diesem Fall wurde das Leben/HP hochgesetzt und verändert die Anzeige unten links.\\
Je nach Ereignis, könen Sie immernoch taktisch entscheiden, ob Sie fliehen, kämpfen, shoppen oder upgraden möchten.

\begin{figure}[htp]
	\centering
	\includegraphics[width=1.00\linewidth]{pics/randomEreignis.png}
	\caption{Zufälliges Ereignis}
	\label{fig1}
\end{figure}

Über Flee gelangen Sie zurück zur Map. Über Fight treten sie unwiederruflich dem Kampf bei, über Shop wird wieder die Möglichkeit geboten zu shoppen, und über Upgrade bieten sich eine Menge Möglichkeiten, Systeme für den kommenden Kampf zu upgraden, siehe Bild.\\
Nach einem Upgrade, besteht immernoch die Möglichketi per Button wieder zurück zur Map zu gelangen.


\begin{figure}[htp]
	\centering
	\includegraphics[width=1.00\linewidth]{pics/upgrade.png}
	\caption{Upgrade der Systeme}
	\label{fig1}
\end{figure}

%%%%%%%%%%%%%%%%%%%%%%%%%%%%%%%%%%%%%%%%%%%%%%%%%%%%%%%%%%%%%%%%%%%%%%%%
\section{Shopping}

Im Shop Screen sind drei Anzeigen zu sehen. Jene oben rechts zeigt an, wie viel Gold/Geld zur verfügung steht. Die am unteren Fensterrand beschreibt das Schiff, in diesem Fall welche Sektion für welche Systeme verantwortlich ist. Die rechte Anzeige beschreibt das Item das darüber angezeigt wird: worum es sich handelt, eigenschaften und dessen Kosten.\\
Über den next Button lässt sich durch das Angebot gucken, die Anzeige darunter, passt sich dementsprechend an.\\
Das Angebot umfasst: Gold, Energie, Crew Member und ein Weapon Lasser. Die Kosten werden entsprechend von dem Gold abgezogen. Secure lässt sich in der vorher upgrade Option kaufen und nicht über den Shop. Möchten Sie einen Crew Member kaufen, so muss die Checkbox der entsprechende Sektion, in die der Crew Member plaziert werden soll, ausgewählt werden, bevor der Kauf über den Buy Button bestätigt werden kann. Möchten Sie den Shop verlassen, dann klicken Sie auf den Back to Map Button in der oberen rechten Ecke. 

\begin{figure}[htp]
	\centering
	\includegraphics[width=1.00\linewidth]{pics/shopScreen.png} %Bild muss noch angepasst werden ohne Sell button
	\caption{Shop}
	\label{fig1}
\end{figure}

%%%%%%%%%%%%%%%%%%%%%%%%%%%%%%%%%%%%%%%%%%%%%%%%%%%%%%%%%%%%%%%%%%%%%%%%
\section{Der Kampf}

\subsection{Beschreibung der Oberfläche}
Im Kampf stehen sich das Raumschiff des Spielers und das des Gegners gegenüber.
Zu Beginn sind die Raumschiffe jeweils von ihrem Schutzschild umgeben.
Die Schrift auf der linken Seite des Fensters beschreibt die Waffen des eigenen Raumschiffs.
In diesem Fall verfügt das Raumschiff des Spielers über die Waffen Rocket Left und Lasse Right.
Diese wiederum haben die Attribute Damage, Bullets und Warmup. Damage gibt an, wie viel Schaden die Waffe dem Gegner bei einem Treffer zufügt. Die Anzahl der Bullets sind
die Anzahl der Schüsse einer Waffe und Warmup gibt die Dauer in Runden an, bis die Waffe wieder einsatzbereit ist.\\

Unten links ist die Energie und dessen Verteilung zu sehen. In dem weißen Balken von links nach rechts: der Betrag des Schutzshcildes, das Leben des Gegners, Secure, Drive und Weapon.
Die grünen Balken stellen die Energie dar, die verteilt werden kann.\\

Die eigenen Sektionen und dessen Eigenschaften sind in roter Schrift unterhalb des gegnerischen Raumschiffs beschrieben: ist die Sektion benutzbar, wie viel Oxygen ist in der Sektion verfügbar, welche Rolle hat die Sektion, dh. für welches System ist sie verantwortlich, ob Health, Weapon oder Engine und welcher Crew Member ist in welcher Sektion.\\

In dem gegnerischen Raumschiff sind drei weiße Symbole zu sehen. Diese repräsentieren die Systeme, die angegriffen werden müssen. In diesem Fall hat das Raumschiff
ein Cockpit, ein Drive System und ein Weapon System.\\

\begin{figure}[htp]
	\centering
	\includegraphics[width=1.00\linewidth]{pics/combatScreenRot01.png}
	\caption{der Kampf}
	\label{fig1}
\end{figure}

\subsection{Spielverlauf}

[...roh..]
Start waiting klicken 2mal , warm up muss runter gehen, bei 0 schießen.
Shießen sektion auswählen, wenn nicht dann pop up mit hinweis Please select a target. tehn you can shoot.
section wählen und mit leertaste schießen. (waffe schießt) wenn kein schuss mehr übrig runden beenden. playing drücken, der gegner reagiert, danach nochmal button um warm zu starten.
dann kann weiter geschossen werden.
Energie muss clever verteielt werden, sonst kann weapon system nicht benutz werden (pop up mit hinweis: Please charge energy to weapon system)
(wird noch geklärt wann eine sektion des gegners kaputt ist). Crew Member können bewegt werden. wenn system kaputt wird automatisch repariert wenn
crew member in sektion, deshalb taktisch hinbewegen.kann an usable erkannt werden wird wieder true gesetzt wenn crew member seinen job gemcht hat.
 energie wir dpro runden neu hinzugefügt. Wechsel der crew member benötigt zeit. erkennt man auch an sanduhr.
reparatur rollenabhängig.Schild: solange da schiff kann nicht beschädigt werden.

Endgegner Befor you travel here you have to visit other planets




%%%%%%%%%%%%%%%%%%%%%%%%%%%%%%%%%%%%%%%%%%%%%%%%%%%%%%%%%%%%%%%%%%%%%%%%
\section{Spielstufen}
Beschreibung der Kamfscreen.
%%%%%%%%%%%%%%%%%%%%%%%%%%%%%%%%%%%%%%%%%%%%%%%%%%%%%%%%%%%%%%%%%%%%%%%%


\subsection{Spiel Unterbrechung}
Beschreibung der Unterbrechung des Spiels und fortsetzen.
%%%%%%%%%%%%%%%%%%%%%%%%%%%%%%%%%%%%%%%%%%%%%%%%%%%%%%%%%%%%%%%%%%%%%%%%


\subsection{Spiel Wiedergeben}
Spielzüge detaliert wiedergeben
\end{document}

%%% Local Variables: 
%%% mode: latex
%%% mode: reftex
%%% mode: flyspell
%%% ispell-local-dictionary: "de_DE"
%%% TeX-master: t
%%% End: 
